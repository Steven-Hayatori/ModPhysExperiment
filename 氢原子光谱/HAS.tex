% !Mode:: "TeX:UTF-8"
% !TEX program  = xelatex
\documentclass[a4paper]{article}
\usepackage{amsmath}
\usepackage{amssymb}
\usepackage{ctex}
%\usepackage{braket}
%\usepackage[european]{circuitikz}
\usepackage{multirow}
\usepackage{float}
\usepackage{graphicx}
\usepackage{geometry}
\geometry{left=2.5cm,right=2.5cm,bottom=2.5cm,top=2.5cm}
\title{近代物理实验报告2.5:氢原子光谱}
\author{林杨\quad 211840092\quad 物理学院}
\date{2024年11月21日}
\begin{document}
\maketitle
\bibliographystyle{unsrt}
%--------main-body------------

\section{引言}
光谱线系的规律与原子结构有内在的联系,因此,原子光谱是研究原子结构的一种重要方法。1885年巴尔末总结了人们对氢光谱测量的结果,发现了氢光谱的规律,提出了著名的巴尔末公式,氢光谱规律的发现为玻尔理论的建立提供了坚实的实验基础,对原子物理学和量子力学的发展起过重要作用.1932年尤里(H. C. Urey)根据里德伯常数随原子核质量不同而变化的规律,对重氢赖曼线系进行摄谱分析,发现氢的同位素——氘的存在.通过巴尔末公式求得的里德伯常数是物理学中少数几个最精确的常数之一,成为检验原子理论可靠性的标准和测量其他基本物理常数的依据。

%WGD-3型光栅光谱仪用于近代物理实验中的氢原子光谱实验,一改以往在大型摄谱仪上用感光胶片记录的方法,而使光谱既可在微机屏幕上显示,由科打印成谱图保存,实验结果准确明了。

\section{实验目的}
\begin{enumerate}
\item 熟悉光栅光谱仪的性能与用法。
\item 用光栅光谱仪测量氢原子光谱巴耳末线系的波长,求里德伯常数。
\end{enumerate}

\section{实验仪器}
在光栅光谱仪中常使用反射式闪耀光栅。锯齿形是光栅刻痕形状.现考虑相邻刻槽的相应点上反射的光线。PQ和P'Q'是以I角入射的光线.QR和Q'R'是以I'角衍射的两条光线。PQR和P'Q'R'两条光线之间的光程差是$s(\sin I+\sin I')$,其中b是相邻刻槽间的距离,称为光栅常数.当光程差满足光栅方程
\begin{equation}
b(\sin I+\sin I') = k\lambda\text{ , k=0,}\pm\text{1, }\pm\text{2, ...}
\end{equation}
时,光强有一极大值,或者说将出现一亮的光谱线。对同一k,根据I、I'可以确定衍射光的波长$\lambda$,这就是光栅测量光谱的原理.闪耀光栅将同一波长的衍射光集中到某一特定的级k上。为了对光谱扫描,将光栅安装在转盘上,转盘由电机驱动,转动转盘,可以改变入射角I,改变波长范围,实现较大波长范围的扫描,软件中的初始化工作,就是改变I的大小,改变测试波长范围。

\section{实验原理}
氢原子光谱是最简单、最典型的原子光谱。用电激发氢放电管(氢灯)中的稀薄氢气(压力在102Pa左右),可得到线状氢原子光谱。瑞士物理学家巴尔末根据实验结果给出氢原子光谱在可见光区域的经验公式
\begin{equation}
\lambda_H = \lambda_0\cfrac{n^2}{n^2-4}
\end{equation}
式中$\lambda_H$为氢原子谱线在真空中的波长。
$\lambda_0$=364.57nm是一经验常数。n取3,4,5等整数。 若用波数表示,则上式变为
\begin{equation}
\tilde{\nu_H} = \cfrac{1}{\lambda_H} = R_H\left(\frac{1}{2^2} - \frac{1}{n^2}\right)
\end{equation}
式中$R_H$称为氢的里德伯常数。
根据玻尔理论,对氢和类氢原子的里德伯常数的计算,得
\begin{equation}
R_Z = \cfrac{2\pi^2me^4Z^2}{(4\pi\varepsilon_0)^2ch^3(1+\frac{m}{M})}
\end{equation}
式中M为原子核质量,m为电子质量,e为电子电荷,c为光速,h为普朗克常数,$\varepsilon_0$为真空介电常数,z为原子序数。 
当M$\to \infty$时,由上式可得出相当于原子核不动时的里德伯常数(普适的里德伯常数):
\begin{equation}
R_{\infty} = \cfrac{2\pi^2me^4z^2}{(4\pi\varepsilon_0)^2ch^3}
\end{equation}
所以
\begin{equation}
R_z = \cfrac{R_{\infty}}{(1+\frac{m}{M})}
\end{equation}
对于氢,有
\begin{equation}
R_H = \cfrac{R_{\infty}}{(1+\frac{m}{M_H})}\label{eq6}
\end{equation}
这里$M_H$是氢原子核的质量。
由此可知,通过实验测得氢的巴尔末线系的前几条谱线的波长,借助(\ref{eq6})式可求得氢的里德伯常数.里德伯常数$R_{\infty}$是重要的基本物理常数之一,对它的精密测量在科学上有重要意义,目前它的推荐值为$R_{\infty}$=10973731.568549(83)$\text{ m}^{-1}$。
\begin{table}[!h]
\centering
\caption{氢的巴耳末线系波长}
\label{table1}
\begin{tabular}{|c|c|}
\hline
谱线符号              & 波长(nm)  \\ \hline
$H_{\alpha}$      & 656.280 \\ \hline
$H_{\beta}$       & 486.133 \\ \hline
$H_{\gamma}$      & 434.047 \\ \hline
$H_{\delta}$      & 410.174 \\ \hline
$H_{\varepsilon}$ & 397.007 \\ \hline
$H_{\zeta}$       & 388.906 \\ \hline
$H_{\eta}$        & 383.540 \\ \hline
$H_{\theta}$      & 379.791 \\ \hline
$H_{\iota}$       & 377.063 \\ \hline
$H_{\kappa}$      & 375.015 \\ \hline
\end{tabular}
\end{table}
表(\ref{table1})为氢的巴尔末线系的波长。值得注意的是,计算$R_H$和$R_{\infty}$时,应该用氢谱线在真空中的波长,而实验是在空气中进行的,所以应将空气中的波长转换成真空中的波长。即$\lambda_{\text{真空}}$ = $\lambda_{\text{空气}}$+$\Delta \lambda_1$,氢巴尔末线系前6条谱线的修正值如表(\ref{table2})所示。
\begin{table}[!h]
\centering
\caption{波长修正值}
\label{table2}
\begin{tabular}{|c|c|c|c|c|c|c|}
\hline
氢谱线                    & $H_{\alpha}$ & $H_{\beta}$ & $H_{\gamma}$ & $H_{\delta}$ & $H_{\varepsilon}$ & $H_{\zeta}$ \\ \hline
$\Delta \lambda_1$(mm) & 0.181        & 0.136       & 0.121        & 0.116        & 0.112             & 0.110       \\ \hline
\end{tabular}
\end{table}

\section{实验内容}
\begin{enumerate}
\item 打开仪器,将氢原子荧光灯贴近接收器的狭缝,打开相应的测量软件,准备测量。
\item 将测量软件的测量分度值调至1nm,测量并记录峰值数据。
\item 将测量软件的测量分度值调至0.5nm,测量并记录峰值数据。
\item 将测量软件的测量分度值调至0.1nm,测量并记录峰值数据。
\end{enumerate}

\section{实验数据}
里德伯常量的理论值约为:10973731.57$\text{m}^{-1}$。
\begin{enumerate}
\item 实验数据\\
三个分度值的测量数据及修正后的真空中的波长数据如下所示:
\begin{enumerate}
\item 1nm
\begin{table}[!h]
\centering
\caption{1nm分度值谱线数据}
\label{table_1nm}
\begin{tabular}{|c|c|c|}
\hline
峰的顺序    &  $\lambda$(nm)  & 真空中的波长$\lambda$(nm) \\ \hline
1   & 410            & 410.181 \\ \hline
2   & 434              & 434.136 \\ \hline
3   & 486             & 486.121 \\ \hline
4   & 657              & 657.116 \\ \hline
\end{tabular}
\end{table}
\item 0.5nm
\begin{table}[!h]
\centering
\caption{0.5nm分度值谱线数据}
\label{table_05nm}
\begin{tabular}{|c|c|c|}
\hline
峰的顺序    &  $\lambda$(nm)  & 真空中的波长$\lambda$(nm) \\ \hline
1   & 410.5            & 410.681 \\ \hline
2   & 434.5             & 434.636 \\ \hline
3   & 486.5             & 486.621 \\ \hline
4   & 656.5              & 656.116 \\ \hline
\end{tabular}
\end{table}
\item 0.1nm
\begin{table}[!h]
\centering
\caption{0.1nm分度值谱线数据}
\label{table_01nm}
\begin{tabular}{|c|c|c|}
\hline
峰的顺序    &  $\lambda$(nm)  & 真空中的波长$\lambda$(nm) \\ \hline
1   & 410.5            & 410.681 \\ \hline
2   & 434.4             & 434.536 \\ \hline
3   & 486.4             & 486.521 \\ \hline
4   & 656.7              & 656.316 \\ \hline
\end{tabular}
\end{table}
\end{enumerate}
\item 计算里德伯常量\\
将3个分度值下得到的数据取平均值,得到表\ref{table_ave}
\begin{table}[!h]
    \centering
    \caption{平均谱线数据}
    \label{table_ave}
    \begin{tabular}{|c|c|c|}
    \hline
    峰的顺序    & 对应的n  & 真空中的波长$\lambda$(nm) \\ \hline
    1   & 6            & 410.514 \\ \hline
    2   & 5             & 434.436 \\ \hline
    3   & 4             & 486.421 \\ \hline
    4   & 3       & 656.849 \\ \hline
    \end{tabular}
    \end{table}

计算得到的4个里德伯常数分别为:10961423.4017, 10964438.8983, 10961119.1566, 10961867.3176。计算平均值,得到实验计算的里德伯常量为10962212.1935,和标准的里德伯常数相比,相对误差为0.105$\%$
\section{误差分析}
\begin{enumerate}
\item 狭缝未对准光源,使实验室的自然光对实验产生干扰。
\end{enumerate}

\end{enumerate}
\section{思考题}
\subsection{氢光谱巴尔末线系的极限波长是多少?}
n趋于无穷时,带入标准里德伯常数,得到$ \lambda_{lim} = 364.506nm$
\subsection{谱线计算值具有唯一的波长,但实测谱线有一定宽度,其主要原因是什么? }
量子力学中,能量和时间存在不确定性关系。实际测量时,总是在一定的时间内得到的平均值。故能量会有不确定性。
\nocite{jiaocai}
\bibliography{ref}
\end{document}